\documentclass[a4paper,10pt]{article}
\usepackage[english]{lsspub} % default is babel's ngerman

% SETUP OF LSSPUB
\lsstitle{How I won Quake III Arena using a tiny sim}%
\lssauthor{Erwin Schmidt}%
\lsstype{Diplomarbeit}%

% CONDITIONALLY SET UP PDF-SPECIFIC STUFF (OPTIONAL)
\usepackage{ifpdf}

\ifpdf
% look at documentation for non-pdflatex setup of hyperref
\usepackage[pdftex]{hyperref} 
\hypersetup{colorlinks=true}
\hypersetup{linkcolor=black}
\hypersetup{citecolor=black}
\hypersetup{pdfauthor=\lsstheauthor}
\hypersetup{pdftitle=\lssthetitle}
\hypersetup{pdfsubject={\lssthetype, Informatik 10, Universität Erlangen-Nürnberg}}
\hypersetup{pdfkeywords={Virtual Reality, Quake III, Ego Shooter}}

% if you want thumbnails for your pdf, you need an additional
% call of thumbpdf and pdflatex after pdflatex converged 
% (i.e. all references were resolved etc.)
%
%\usepackage{thumbpdf}
%\hypersetup{pdfpagemode=UseThumbs}
\fi

% START LATEX DOCUMENT BODY

\begin{document}

% The following command creates the title page(s) and the text required 
% by the Pruefungsamt, stays German even if english option enabled
% You can optionally provide a date for the signature line),
% otherwise \today is used.
\makelssthesis{Prof.~Dr.~U.~Rüde}{Dipl.-Inf.~M.~Kowarschik}{1.2.1999 -- 31.12.2000}

% Now the acutal thesis can start
\section{Introduction and Conclusions}

The use of a tiny sim in \emph{Quake III Arena} offers many advantages.
The most important being, that one is harder to ``frag'', as can be seen in
Tab.~\ref{FRAGS}.

\begin{table}[h]
  \centering
  \begin{tabular}{|l|*{4}{c}|}
    \hline
    \textbf{Sim Name:} & War Dog & Neanderthaler & Prime Evil & Erwin Schmidt \\
    \hline
    \textbf{Size:} & 2'' & 5'' & 7'' & 1.5'' \\
    \hline
    \textbf{Frag Rate:} & 35\%  & 45\% & 95\% & 2\% \\ 
    \hline
  \end{tabular}
  \caption{Frag rate depending on sim size}
  \label{FRAGS}
\end{table}

\end{document}
